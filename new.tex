


\documentclass[runningheads,a4paper]{llncs}

\usepackage{amssymb}
\setcounter{tocdepth}{3}
\usepackage{graphicx}

\usepackage{url}

\newcommand{\keywords}[1]{\par\addvspace\baselineskip
\noindent\keywordname\enspace\ignorespaces#1}

\begin{document}

\mainmatter  % start of an individual contribution

% first the title is needed
\title{A Proxemic Multimedia Interaction over the Internet of Things}

% a short form should be given in case it is too long for the running head
\titlerunning{A Proxemic Multimedia Interaction over the Internet of Things}

% the name(s) of the author(s) follow(s) next
%
% NB: Chinese authors should write their first names(s) in front of
% their surnames. This ensures that the names appear correctly in
% the running heads and the author index.
%
\author{Ali Danesh, Mukesh Saini, and Abdulmotaleb El Saddik }

% (feature abused for this document to repeat the title also on left hand pages)

% the affiliations are given next; don't give your e-mail address
% unless you accept that it will be published




%
% NB: a more complex sample for affiliations and the mapping to the
% corresponding authors can be found in the file "llncs.dem"
% (search for the string "\mainmatter" where a contribution starts).
% "llncs.dem" accompanies the document class "llncs.cls".
%

\toctitle{Lecture Notes in Computer Science}
\tocauthor{Authors' Instructions}
\maketitle


\begin{abstract}

With the rapid growth of online devices, a new concept of Internet of Things (IoT) is emerging in which everyday devices will be connected to the Internet. As the number of devices in IoT is increasing, so is the complexity of the interactions between user and devices. There is a need to design intelligent user interfaces that could assist users in interactions. The present study proposes a proximity-based user inter- face for multimedia devices over IoT. The proposed method employs a cloud-based decision engine to support user to choose and interact with the most appropriate device, reliving the user from the burden of enu- merating available devices manually. The decision engine observes the multimedia content and device properties, learns user preferences adap- tively, and automatically recommends the most appropriate device to interact. The system evaluation shows that the users agree with the pro- posed interaction 70% of the times. 

\emph{abstract} environment.
\keywords{Proxemic interaction; multimedia interaction ; user inter- face; elicitation study 
}
\end{abstract}


\section{Introduction}

The Internet has been expanding very rapidly with time. Now, more than 2.7 billion people (almost 39 of world’s population) have access to it and use it in their daily life [1]. The number of devices that are connected to the Internet has been growing dramatically. Therefore, a new concept of Internet of Things (IoT) is emerging [2]. More than 11.2 billion devices were connected to the Internet in 2013 and it is predicted that there will be around 50 billion devices online by 2020 [3]. These uniquely identifiable objects and devices move and interact with each other to accomplish various tasks. In other words, IoT is like a big and dynamic society of objects and people. But we are far from the ubiquitous computing vision of Weiser [4] due to two main reasons: the lack of an appropriate task- centered User Interface (UI) design approach and the lack of reliable support for distributable user interfaces in ubiquitous environments [5]. Thus, there is a need for a new generation of specifically designed user interfaces for IoT.
\newline
More than 7 billion people constantly use verbal and nonverbal communica- tion techniques to interact with their society all across the world. Verbal com- munication is considered as the main channel since it is used explicitly. However, 
\newpage


nonverbal communication such as Proxemics, Haptics, Body Language, etc. plays an important role in the human society because it uses implicit interactions. In the same way, nonverbal communication can also enhance the interactions within the IoT, which is ike human’s society except the size of IoT is few times bigger than human society. Thus, nonverbal techniques should be used in order to build proper user interfaces for ubiquitous environments, particularly the IoT. 
\newline
\newline
The present study proposes a distributed user interface that provides a suit- able environment for multimedia interactions over the IoT. It consists of a cloud- based decision engine that manages the proxemic interactions. This engine is called Proxemic Interaction Unit (PIU). PIU uses multimedia devices within the IoT as elements of a universal UI in order to assist users to make use of their surrounding devices. We collected some of the effectual multimedia inter- action variables such as device properties, multimedia content attributes, user’s preferences, etc. to propose a scoring mechanism for the PIU. A group of these variables are obtained by conducting a user survey and elicitation study while the rest are extracted from the literature. To further personalize user preferences, we update the variables after every interaction. The evaluation shows that 70  of the times the user interface chooses the same device a user would. We reckon that the user preference learning mechanism will further increase the accuracy of the user interface. 
\newline
The rest of this paper firstly presents a review of related works in Section 2. Next, Section 3 describes the architecture and details of proposed UI. The results of an evaluation user study are given in Section 4, which is followed by brief discussion in Section 5. This study ends with the conclusion and future work in Section 6. 





\section{Related Work }

Researchers have been trying to develop new UIs over the IoT for the last few years. However, the idea of using proxemic interactions has been around for a while. Hall highlighted the influence of proxemic behavior on interpersonal communication in 1966 [6]. He divided proxemic interactions into two levels: micro-level, which studies the way people interact with each other in daily life and macro-level, which reviews the space organization of houses, buildings and ultimately towns. More recently, Greenberg et al. proposed a practical version of proxemic interaction that considers people, digital devices, and non-digital objects [7]. They defined five dimensions for proxemic interaction: distance, ori- entation, movement, identity, and location. Every measured change in any di- mension can trigger an interaction. Marquardt et al. used this terminology to develop a proximity toolkit in order to aid fast prototyping of proxemic inter- actions [8]. Despite the fact that researchers have been trying to define a more precise structure and terminology for proxemic interactions recently, proxemics have been involved in applications for a long time. 
\newline
\newpage

Researchers study the interactions between smart objects in the IoT (e.g. [9]). However, the research community has paid more attention to the human-device interactions (e.g. [10,11]). Ju et al. proposed the Range as a public interactive whiteboard, which supports co-located, ad-hoc meetings [12]. It uses the prox- imity sensing to proactively manage the transitions between display and authors (e.g. to clear the space for writing). Wang et al. introduced Proxemic Peddler, which is a public display [13]. This public display can capture and preserve the attention of pedestrians. Both of the aforementioned studies are examples of human-device proxemic interactions in micro-level where the proximity space is very small (e.g. one room). Researchers also have developed some proximity aware applications at macro-level (e.g. multiple rooms). The active badge, which was introduced by Want et al., is one of them [14]. They embedded an infrared beacon into their badge in order to connect it to a sensory network. Using this architecture, they could track employees in the office and redirect their phone calls to their current stations. Active badge was successfully implemented and used in large scale. 
\newline
Moreover, there are some studies that focus on possible human-device prox- emic interactions in households. The EasyLiving is one of the earliest projects in this group [15]. It emphasizes on an architecture, which can connect different devices together in order to enhance the user experience. Ramani et al. proposed a new location tracking system for media appliances in the wireless home net- works, which can be used to build a proxemic media redirection system [16]. Recently, Sørensen et al. introduced and evaluated the AirPlayer, which is a multi-room music system that uses the proxemic interaction [17]. \newline
In conclusion, there is no custom designed proxemic interface for multimedia interactions. The proposed UI is designed and optimized for proxemic multime- dia interaction, which makes it a task-specific UI. Furthermore, we believe that a completely distributed solution (e.g. [17]) cannot meet the growing require- ments of the IoT users. Therefore, the proposed method employs a cloud-based decision engine (PIU), which considers additional information such as multi- media content properties, user’s preferences, devise capabilities etc. in order to improve multimedia interactions. Moreover, the current trend is that a cloud- base database keeps the shared resources including multimedia data. Hence, our design of a cloud-based decision engine is in accordance with the current tech- nological trends





\end{document}
